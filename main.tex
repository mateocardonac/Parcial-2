\documentclass{article}
\usepackage[utf8]{inputenc}
\usepackage[spanish]{babel}
\usepackage{listings}
\usepackage{graphicx}
\graphicspath{ {images/} }
\usepackage{cite}

\begin{document}

\begin{titlepage}
    \begin{center}
        \vspace*{1cm}
            
        \Huge
        \textbf{}
            
        \vspace{0.5cm}
        \LARGE
        Parcial 2
        \\
        Primera Parte
            
        \vspace{1.5cm}
            
        \textbf{Mateo Cardona Correa}
            
        \vfill
            
        \vspace{0.8cm}
            
        \Large
        Despartamento de Ingeniería Electrónica y Telecomunicaciones\\
        Universidad de Antioquia\\
        Medellín\\
        Septiembre de 2021
            
    \end{center}
\end{titlepage}

\tableofcontents
\newpage
\section{Introduccion al Trabajo}\label{intro}
El trabajo aqui presentado se realiza bajo la consideracion de la presentacion de un problema planteado en la normativa de la cotidianidad, presentando un problema tan comun como puede ser el desarrollo de un programa para transformar nociones, conceptos, programas y algoritmos digitales a un ambiente no digital, como puede ser una pantalla o un grupo de luces, en este caso, leds.

\subsection{Analisis del problema}\label{}
El problema planteado para la realizacion del ejercicio se basa en una serie de dos grandes problemas: \\
-El escaneo eficiente de imagenes digitales de cualquier tipo de tamaño o variable. \\
-La creacion de un circuito tipo RGB que sea capaz de pasar de una imagen digital a panel de luces Led capaz de interpretar y recrear la imagen anteriormente mencionada
\subsection{Variantes a considerar}\label{}
Las variantes mas importantes a considerar son:\\
-Creacion de un algoritmo basico capaz del escaneo y rediseño de imagenes mas grandes y complejas en imagenes mas pequeñas y sencillas\\
-Implementacion del algoritmo capaz de transformar la imagen mas pequeña en una serie de instrucciones y señales faciles de leer para cualquier maquina\\
-La creacion de un circuito capaz de leer las instrucciones y señales para poder manejar un tablero de tipo luces LED RGB de magnitudes cercanas al 8x8
\newpage
\section{Esquema} \label{contenido}
Aqui se abordara un esquema basico de trabajo: \\
-Paso 1: Transformacion de imagenes en codigo\\
-Paso 2: Transformacion de codigo a señales (Del codigo de QT a un codigo que un Arduino en Tinkercad sea capaz de leer e interpretar)\\
-Paso 3: Transformacion de señales (codigo de Arduino) en una ejecucion de un tablero de luces Led RGB
\newpage

\section{Algoritmo/Codigo} \label{algoritmo}
A continuación, se presentara un algoritmo código creado como base para el desarrollo de actividad:\\
-Imagen Base\\
-Transformar imagen base\\
\phantom{---}-Imagen mas pequeña\\
\phantom{---}-Transformar Imagen mas pequeña en codigo\\
-Guardar codigo en archivo .txt\\
-Pasar el codigo (archivo .txt) a un circuito de Tinkercad\\
-Lectura del codigo\\
-Transformacion Lectura a señales electricas\\
-Pasar de las señales electricas al tablero RGB\\
-Que el tablero RGB muestre la imagen mas pequeña\\

Este seria un algoritmo algo basico para la creacion del codigo para el desarrollo de la actividad
\newpage

\section{Consideraciones} \label{consideraciones}
Para la creacion y la implementacion del algoritmo se tienen una serie de consideraciones a tener en cuenta:\\
-Crear un codigo funcional tanto en Tinkercad como en QT\\
-Que el codigo sea capaz de "leer" una imagen cualquiera\\
-Que sea capaz de transformar esta imagen en una imagen muchisimo mas pequeña\\
-Crear un circuito capaz de mostrar esta imagen mas pequeña\\
-Que el circuito utilice un esquema de leds RGB\\
-Que sea lo suficientemente eficiente para manejar un tablero de leds bastante grande\\
Estas serian algunas de las consideraciones a tener en cuenta para el desarrollo de la actividad
\end{document}
